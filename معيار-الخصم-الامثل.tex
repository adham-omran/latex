\documentclass[a4paper]{article}%
\usepackage{fontspec}
\usepackage{pgfplots}
\usepackage{amsmath}
\usepackage{derivative}
\usepackage{tikz}
% \usepackage{bidi}
\usepackage{multicol}
\usepackage{polyglossia, bidi}

\setmainlanguage[locale=mashriq]{arabic}
\newfontfamily\arabicfont[Script=Arabic]{ScheherazadeNew-Regular} % has to match
% .otf or .ttf name

\def\matharabic#1{\ensuremath{\text{\begin{Arabic}#1‎\end{Arabic}}}}

\setotherlanguage{english}
% \title{حساب فترة الاسترداد بطريقة (الخصم المتتابع)}
\title{معيار سعر الخصم الامثل \textenglish{Internal Rate of Return (IRR)}}
\author{}
\date{2022/04/25}

\pgfplotsset{grid style={dashed,gray}}
% \pgfplotsset{minor grid style={dashed,red}}
% \pgfplotsset{major grid style={dotted,green!50!black}}


\begin{document}
\maketitle

هو سعر الفائدة الذي يجعل صافي القيمة الحالية مساوي الى الصفر ($\text{NPV} = 0$) أو هو
اعلى معدل للعائد الذي يحصل عليه المشروع الاستثماري عن نشاطه الجاري ويحسب
بطريقتين:

\section{الطريقة الرياضية}%

\begin{align*}
  IRR = r_1 + \frac{(r_{2}-r_{1}) \cdot \text{NPV}_{+}}{|\text{NPV}_{+}+\text{NPV}_{-}|}
\end{align*}
حيث \\
$r_1$: سعر الخصم الأدنى \\
$r_2$: سعر الخصم الأعلى \\
$\text{NPV}_{+}$: صافي القيمة الحالية الموجبة وصيغته $NPV_{+} = \sum PV_{1} - I_{o} > 0$ \\
$\text{NPV}_{-}$: صافي القيمة الحالية السالبة وصيغته $NPV_{-} = \sum PV_{2} - I_{o} < 0$ \\

مثال: في مشروع اقتصادي اذا علمت:
\[r_{1} = 0.20, r_{2} = 0.30, I_o = 94, \sum PV_1 = 100, \sum PV_{2} = 90\]
المطلوب: ايجاد سعر الخصم الامثل ($IRR$) رياضياً وبيانياً

\begin{align}
  \text{NPV}_+ = \sum PV_{1} - I_{o} \\
  \text{NPV}_+ = 100 - 94 = 6 > 0 \\
  \text{NPV}_- = \sum PV_2 - I_{0} \\
  \text{NPV}_- = 90 - 94 = -4 < 0 \\
  \text{IRR} = \frac{(r_{2}-r_{1}) \cdot \text{NPV}_{+}}{|\text{NPV}_{+}+\text{NPV}_{-}|} \\
  \text{IRR} = 0.20 + \frac{(0.30 - 0.20) \cdot 6}{|6+4|} \\
  \text{IRR} = 0.20 + 0.06 = 0.26
\end{align}
ملاحظة: سعر الخصم الامثل ($\text{IRR}$) اكبر من سعر الخصم الادنى واصغر من سعر الخصم الاعلى, اي ان:
\[r_1 = 0.20 < \text{IRR} = 0.26 < r_2 = 0.30\]

\section{الطريقة البيانية}%

\begin{tikzpicture}

  \begin{axis}[
    xlabel=IRR,
    ylabel=NPV,
    width=10cm,height=9cm,
    % axis lines = middle
    % axis line style={draw=none}
    % legend style={at={(0.0,.91)},anchor=west}
    xtick = {0, 0.1, 0.2, 0.3, 0.4},
    xmin = 0,
    xmax = 0.4,
    grid=both
    ]

    % Add values and attributes for the first plot
    % x-axis
    \addplot[color=black] coordinates {
      (0,0) (0.4,0)
    };


    \addplot[color=black] coordinates {
      (0,6) (0.2, 6)
      (0.2, 6) (0.2,0)
    };

    \addplot[color=blue, dashed] coordinates {
      (0.2,6) (0.3, -4)
    };

    \addplot[color=black] coordinates {
      (0,-4) (0.3, -4)
      (0.3, -4) (0.3,0)
    };

    \node[label={above right:{$A (0.20,6)$}},circle,fill,inner sep=2pt] at (axis
    cs:0.2,6) {};

    \node[label={above right:{$C (0.26,0)$}},circle,fill,inner sep=2pt] at (axis
    cs:0.26,0) {};

    \node[label={above right:{$B (0.30,-4)$}},circle,fill,inner sep=2pt] at (axis
    cs:0.3,-4) {};
    % \legend{Case 1,Case 2}
  \end{axis}
\end{tikzpicture}

الفكرة والاستنتاج:
\begin{enumerate}
  \item يتم رسم الاحداثيات العمودي (\textenglish{NPV}) الجزء الموجب لغاية 6 والجزء السالب لغاية $-4$ والمحور الافقي (\textenglish{IRR}).
  \item يتم تحديد النقاط (التوليفة الهندسية) في الجزء الموجب ($0.20, 6$) \textenglish{A} وفي الجزء السالب ($0.30, -4$)\textenglish{B}.
  \item يتم رسم خط مستقيم (المتقطع) ليصل بين النقطتين ($A, B$).
  \item يقطع المحور الافقي ($\text{IRR}$) عند النقطة $C$ واحداثياتها ($0.26, 0$) \textenglish{C} والتي تمثل سعر الخصم الامثل ($\text{IRR} = 0.26$).
  \item يجعل \textenglish{IRR} الامثل صافي القيمة الحالية مساوي الى الصفر ($\text{NPV} = 0$)
\end{enumerate}

الاستنتاج: نفس الاستنتاج بالطريقة الرياضية وهو ان:
\[r_1 = 0.20 < \text{IRR} = 0.26 < r_2 = 0.30\]
\end{document}
