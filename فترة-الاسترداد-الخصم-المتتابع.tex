\documentclass[a4paper]{article}%
\usepackage{fontspec}
\usepackage{amsmath}
\usepackage{derivative}
\usepackage{tikz}
% \usepackage{bidi}
\usepackage{multicol}
\usepackage{polyglossia, bidi}

\setmainlanguage[locale=mashriq]{arabic}
\newfontfamily\arabicfont[Script=Arabic]{ScheherazadeNew-Regular} % has to match
% .otf or .ttf name

\def\matharabic#1{\ensuremath{\text{\begin{Arabic}#1‎\end{Arabic}}}}

\setotherlanguage{english}
% \title{حساب فترة الاسترداد بطريقة (الخصم المتتابع)}
\title{محاضرة تطبيقي}
\author{}
\date{2022/04/21}

\begin{document}
\maketitle
\section{حساب فترة الاسترداد بطريقة (الخصم المتتابع)}
في هذه الطريقة يتم خصم (او طرح) الارباح المتحققة خلال سنوات عمر المشروع ($n$) من
الكلفة الاستثمارية الاساسية ($I_{o}$) الى المدة التي يكون فيها الفرق مساوي الى
(الصفر).
\\
مثال: في مشروع اقتصادي معين, اذا علمت ان $I_{o} = 100$ وحدة نقدية والارباح
المتحققة خلال عمر المشروع ($5$) سنوات هي على التوالي
($30,25,20,15,10$) وحدة نقدية

المطلوب: ايجاد فترة الاسترداد (pbp) بطريقة الخصم المتتابع


\begin{english}

  \begin{table}[htb]
    \centering
    \begin{tabular}{|c|c|c|}
      $N$ & $\pi_{i}$ & $I_{o}-\pi_{i}$ \\
      \hline
      $1$ & $10$ & $100-10=90$ \\
      $2$ & $15$ & $90-15=75$  \\
      $3$ & $20$ & $75-20=55$  \\
      $4$ & $25$ & $55-25=30$  \\
      $5$ & $30$ & $30-30=0$
    \end{tabular}
  \end{table}
\end{english}

فترة الاسترداد (5) سنوات لان الفرق بين الكلفة الاستثمارية الاساسية ($I_{o}$)
والارباح المتحققة ($\pi_{i}$) مساوي الى الصفر عند السنة الخامسة ($n = 5$)
والمشروع ذو جدوى اقتصادية

ملاحظة: احياناً يتم تحديد فترة فاصلة لاسترداد التكاليف من قبل المستثمر
(\textenglish{Cut-off period}) فاذا كانت في هذا المثال مثلاً ($n = 3$) تقول ان
المشروع ليس ذو جدوى اقتصادية لان الفترة الفاصلة لاسترداد التكاليف الاستثمارية من
قبل المستثمر وهي ($3$) سنوات لا يمكن تحقيقها.

\section{المعايير التي تعتمد على سعر الخصم}


\subsection{معيار اجمالي القيمة الحالية}%
هو حاصل ضرب(الارباح الصافية ($\pi_{i}$) $\times$ معامل الخصم ($RC$)) والمشروع
يكون ذي جدوى اقتصادية عندما يكون ($\sum PV_i > 1$)

مثال: توفرت لديك الارباح الصافية (التدفق النقدي) لمشروع اقتصادي معين خلال ($5$)
سنوات وبسعر خصم ($0.1$) والمطلوب ايجاد اجمالي القيمة الحالية
($\sum_{i=1}^{5}PV_{i}$) وهل المشروع ذو جدوى اقتصادية اذا علمت ان الكلفة
الاستثمارية الاساسية هي ($I_{o} = 100$) وحدة نقدية و ($\pi_{i}$) على التوالي هي
($50,40,30,20,10$) وحدة نقدية

\begin{english}
  \begin{table}[htpb]
    \begin{tabular}{c|c|c|c}
      $n$ & $RC = \frac{1}{(1+r)^{n}}$ & $\pi_{i}$ &
                                                     $PV_{i} = RC \times \pi_{i}$ \\
      \hline
      1 & $RC_{1} = \frac{1}{(1+0.10)^{1}} = 0.909$ & 10 & $0.909 \times 10 = 9.09$  \\
      2 & $RC_{2} = RC_{1} \times RC_{1} = 0.826$ & 20 & $0.826 \times 20 = 16.52$ \\
      3 & $RC_{3} = RC_{2} \times RC_{1} = 0.751$ & 30 & $0.751 \times 30 = 22.53$ \\
      4 & $RC_{4} = RC_{3} \times RC_{1} = 0.683$ & 40 & $0.683 \times 40 = 27.32$ \\
      5 & $RC_{5} = RC_{4} \times RC_{1} = 0.62$ & 50 & $0.62 \times 50 = 31$
      \\
      \hline
          & & & $\sum_{i=1}^{5}PV_{i} = 106.46 > 1$
    \end{tabular}
  \end{table}
\end{english}

\newpage
المشروع ذو جدوى اقتصادية
\subsection{معيار صافي القيمة الحالية \textenglish{Net Present Value (NPV)}}%
هو الفرق بين اجمالي القيمة الحالية والكلفة الاستثمارية الاساسية والمشروع ذو جدوى اقتصادية عندما يكون الفرق اكبر من او يساوي الصفر
\\

في المثال السابق

\begin{align}
  NPV = \sum PV_i - I_o \ge 0 \\
  = 106.46 - 100 = 6.46 \ge 0
\end{align}
المشروع ذو جدوى اقتصادية
\subsection{معيار انتاجية النفقة الاستثمارية أو (معيار المنافع الى التكاليف)}
هو حاصل قسمة اجمالي القيمة الحالية على الكلفة الاستثمارية الاساسية, المشروع ذو
جدوى اقتصادية عندما يكون حاصل القسمة أكبر أو مساوي الى الواحد.
\setcounter{equation}{0}
في المثال السابق
\begin{align}
  (\frac{B}{C}) = \frac{\sum PV_{i}}{I_{o}} \ge 1\\
  % Benefit / Cost
  (\frac{B}{C}) = \frac{106.46}{100} \\
  (\frac{B}{C}) = 1.064 \ge 1
\end{align}
المشروع ذو جدوى اقتصادية

\end{document}
