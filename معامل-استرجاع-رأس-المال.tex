\documentclass[a4paper]{article}%
\usepackage{fontspec}
\usepackage{pgfplots}
\usepackage{amsmath}
\usepackage{derivative}
\usepackage{tikz}
% \usepackage{bidi}
\usepackage{multicol}
\usepackage{polyglossia, bidi}

\setmainlanguage[locale=mashriq]{arabic}
\newfontfamily\arabicfont[Script=Arabic]{ScheherazadeNew-Regular} % has to match
% .otf or .ttf name

\def\matharabic#1{\ensuremath{\text{\begin{Arabic}#1‎\end{Arabic}}}}

\setotherlanguage{english}
% \title{حساب فترة الاسترداد بطريقة (الخصم المتتابع)}
\title{معامل استرجاع رأس المال المستثمر ($\bar{RC}$)}
\author{}
\date{2022/04/28}

\begin{document}
\maketitle

هي الفترة الزمنية اللازمة لاسترجاع رأس المال المستثمر وتمثل نسبة مئوية لحصة
استرجاع السنة الواحدة (من عمر المشروع الافتراضي) من الكلفة الأولية الاستثمارية
(الاساسية) وتحسب رياضياً:
\[\bar{RC} = \frac{r(1+r)^{n}}{(1+r)^{n}-1}\]
حيث أن: \\
$r$: سعر الخصم (الفائدة) \\
$n$: سنوات (عمر) المشروع \\

مثال: توفرت لديك البيانات الاتية عن مشروع اقتصادي حيث:
\[r = 0.07,\ n=10\]
المطلوب: ايجاد \(\bar{RC}\)

\[\bar{RC} = \frac{0.07(1+0.07)^{10}}{(1+0.07)^{10}-1} = \frac{0.07(1.967)}{(1.967)-1} = 0.1424\]
أي أن المشروع يسترد سنوياً ($14.24\%$) من الكلفة الاستثمارية الاساسية وهو يمثل
ايضاً العائد على رأس المال المستثمر ($\text{RoC}$)

ملاحظة: في حالة اضافة معلومة الكلفة الاستثمارية الاساسية ($I_{o} = 200$) الى
السؤال, يصبح المطلوب:
\begin{enumerate}
  \item ايجاد $\bar{RC}$
  \item ايجاد $\text{RoC}$
  \item مقدار الكلفة المستردة سنوياً
  \item فائض الربح المتحقق
  \item مقدار القسط السنوي للاسترداد
  \item فترة الاسترداد؟
\end{enumerate}
 تحقق من الحل؟
\begin{align}
  \bar{RC}=0.1424 \\
  RoC = 14.24\%  \\
  0.1424\times 200 = 28.48\ \matharabic{وحدة نقدية (الكلفة المستردة سنوياً)}\\
  (28.48\times 10) - 200 = 84.8\ \matharabic{وحدة نقدية فائض الربح المتحقق} \\
  \frac{84.8}{28.48}=2.977\ \matharabic{وحدة نقدية مقدار القسط السنوي للاسترداد} \\
  10 - 2.977 = 7.023\ \matharabic{سنة فترة الاسترداد}
\end{align}
التحقيق: \[\text{RoC} = \frac{1}{\text{pbp}} = \frac{1}{7.023} = 0.1424\]

\end{document}
